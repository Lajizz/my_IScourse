\documentclass[a4paper, 12pt]{article}
\usepackage{ctex}
\usepackage{amsmath}
\usepackage{amsthm,amsmath,amssymb}
\usepackage{mathrsfs}

\title{Chapter 3: 信道与信道容量}
\author{Xuan}
\begin{document}
    \maketitle
    \section{信道的基本概念}
    \subsection{二进制离散信道(BSC)}
    该信道模型的输入和输出信号的符号数都是2,即$X\in A={0,1}$和$Y\in B={0, 1}$,转移概率为
    \begin{equation}
        \begin{aligned}
            p(Y=0|X=1)&=p(Y=1|X=0)=p\\
            p(Y=1|X=1)&=p(Y=0|X=0)=1-p
        \end{aligned}
    \end{equation}
    \subsection{加性高斯白噪声信道(AWGN)}
    \[Y=X+G\]
    G是一个零均值、方差为$\sigma^2$的高斯随机变量,当$X=a_i$给定后,Y是一个均值为$a_i$、方差为$\sigma^2$的高斯随机变量
    \[p_Y(y|a_i)=\frac{1}{\sqrt{2\pi\sigma^2}}e^{-(y-a_i)^2/2\sigma^2}\]
    \section{信道}
    信道可以看成是转移概率

    对于信息$M\in\mathcal{M}$,传输速率$R=\frac{log(\mathcal{M})}{n}$,信道传输的过程即可表示为:
    \[M\rightarrow x^n(M)\xrightarrow{p(y|x)}y^n \rightarrow \hat{M}\]
    
    \begin{enumerate}
        \item 设计一个方案,该方案可以达到某传输概率
        \item 证明超出该传输速率无法传输 $\iff$ 能够传输的都不超过这个速率
    \end{enumerate}
    \subsection{1.设计方案}
    \subsubsection{典型序列}
    Define: $x^n$ is (n, $\varepsilon$) typical, when $|N(x|x^n)-p(x)|<\varepsilon n$, for all
    $x\in \mathcal{X}$, where $N(x|x^n)$ is the empirical distribution.

    $Pr(x^n is typical)\rightarrow 1$, when n $\rightarrow \infty$, which can be proved by Law of Large Numbers
    \subsection{典型集}
    Set $T(n, \varepsilon)$ 为典型序列的集合

    \begin{enumerate}
        \item $Pr(x^n \in T(n, \varepsilon))\rightarrow 1$
        \item $|T(n,\varepsilon)|\approx2^{nH(x)}$
        \subitem 2.1 $x^n\in T(\varepsilon, n), p(X^n=x^n)\approx 2^{-nH(x)}$
    \end{enumerate}
    \subsubsection{Proof}
    \[
        p(x^n)=\prod_{i=1}^{n}p(x_i)=\prod_{x\in \mathcal{X}}p(x)^{N(x|x^n)n} (*)
    \]
    $\Rightarrow$所有典型序列的概率都差不多大
    \[
        log(*)=\sum_{x\in \mathcal{X}}nN(x|x^n)logp(x)\approx n\sum_{x\in \mathcal{X}}p(x)logp(x)=-nH(x)
    \]
    Pr($x^n$: $x^n$ is typical) = * $\approx 2^{-nH(x)}$

    \subsection{典型集和散度的关系}
    $Pr(x^n\in T(n, \varepsilon))=?$
    
    $x^n\in T(n, \varepsilon), N(x|x^n)~p(x)$

    \begin{equation}
        \begin{aligned}
            Pr(x^n)&=\prod q(x)^{nN(x|x^n)}\\
            &\approx \prod q(x)^{np(x)}\\
            &=2^{log\prod q(x)^{np(x)}}\\
            &=2^{np(x)\sum logq(x)}\\
            &=2^{-np(x)log\frac{1}{q(x)}}
        \end{aligned}
    \end{equation}
    \begin{equation}
        \begin{aligned}
            Pr(T(n, \varepsilon)) &= \sum_{x^n \in T(\varepsilon, n)}Pr(x^n)\\
            &\approx 2^{-np(x)log p(x)}*2^{-np(x)log \frac{1}{q(x)}}\\
            &=2^{-nD(p||q)}
        \end{aligned}
    \end{equation}
    \subsection{条件典型集}
    典型条件集的大小:$2^{nH(Y|X)}$    
\end{document}