\documentclass[a4paper, 12pt]{article}
\usepackage{ctex}
\usepackage{amsmath}
\title{Chapter 2}
\author{Xuan}
\begin{document}
    \maketitle
    \section{离散信源熵与互信息}
    \subsection{信源熵}
    $H(X)=\sum_ip(x_i)\log_2\frac{1}{p(x_i)}=-\sum_ip(x_i)\log_2p(x_i)$
    \paragraph{性质:}
    \begin{enumerate}
        \item $H(X)\ge0$
        \item $H(X)\le\log_2|X|$
    \end{enumerate}
    当某一符号$x_i$的概率为$p_i$为零时,$p_i\log_2p_i$在熵公式中无意义,为此规定这时的$p_i\log_2p_i$为零。
    当信源$X$中只含一个符号$x$时,必定有$p(x)=1$,此时信源熵$H(x)$为零,是确定信源。
    \subsubsection{证明$H(x)\le\log_2|X|$:}
    \paragraph{方法一:} 求偏导,略
    \paragraph{方法二:} 利用相对熵\\
    假设q服从均匀分布,即$q_i=\frac{1}{|X|}$
    \begin{align}
        D(p||q)&=\sum_ip_i\log_2\frac{p_i}{q_i}\\
        &=\sum_ip_i\log_2p_i-\sum_ip_i\log_2q_i\\
        &=-H(x)-\sum_ip_i\log_2q_i\ge0\\
        &=-H(x)+\sum_ip_i\log_2|X|\ge0
    \end{align}
    所以我们可以得到
    \[H(x)\le\log_2|X|\]
    \subsection{相对熵}
    p相对于q的相对熵定义为
    \[D(p||q)=\sum_ip_i\log_2\frac{p_i}{q_i}\]
    相对熵也成为交叉熵或Kullback-Leibler距离(KL距离)。它满足两个要求:
    \begin{enumerate}
        \item 非负性
        \item 当且仅当对所有i,$p_i=q_i$时,相对熵为零
    \end{enumerate}
    \subsubsection{证明$D(p||q)\ge0$}
    \paragraph{下凸函数的性质} $pf(x_1)+(i-p)f(x_2)\ge f(px_1+(1-p)x_2)$
    \paragraph{Jensen's Inequality:} 对于下凸函数而言,
    \[Ef(X)\ge f(EX)\]
    上凸函数反之。
    \begin{enumerate}
        \item $|X|=2$时,由下凸函数性质可证
        \item Jensen's Inequality
    \end{enumerate}
    下面先证Jensen's Inequality:\\
    Assume $|X|=k-1$, Jensen's Inequality holds\\
    Prove $|X|=k$, Inequality holds as well.\\
    Assume $\sum_{i=1}^{k-1}p(x_i)=q=1-p(x_k)$
    \begin{align}
        Ef(X)&=p(x-k)f(x_k)+\sum_{i=1}^{k-1}p(x_i)f(x_i)\\
        &=p(x_k)f(x_k)+q\sum_{i=1}^{k-1}\frac{p(x_i)}{q}f(x_i)\\
        &=p(x_k)f(x_k)+(1-p(x_k))f(\sum_{i=1}^{k-1}\frac{p(x_i)}{1-p(x_k)}x_i)\\
        &\ge f(p(x_k)x_k+\sum_{i=1}^{k-1}p(x_i)x_i)\\
        &=f(EX)
    \end{align}
    下面证$D(p||q)\ge0$\\
    \begin{align}
        -D(p||q)&=-\sum_{x\in A}p(x)\log_2\frac{p(x)}{q(x)}\\
        &=\sum_{x\in A}p(x)\log_2\frac{q(x)}{p(x)}\\
        &\le \log_2\sum_{x\in A}p(x)\frac{q(x)}{p(x)}\\
        &=\log_2\sum_{x\in A}q(x)\\
        &\le \log_2\sum_{x\in Z}q(x)\\
        &=\log_21\\
        &=0
    \end{align}
    \subsection{联合熵、条件熵}
    \paragraph{条件熵} 
    \begin{align}
        H(X|Y)&=\sum_jp(y_j)H(X|y_j)\\
        &=\sum_{ij}p(y_j)p(x_i|y_j)I(x_i|y_j)\\
        &=\sum_{ij}p(x_i,y_j)\log_2p(x_i|y_j)
    \end{align}
    \paragraph{联合熵}
    \[H(X,Y)=\sum_{ij}p(x_i,y_j)I(x_i,y_j)=-\sum p(x_i,y_j)\log_2p(x_i,y_j)\]
    \[H(X|Y)=\sum p(y)H(X|y)\]
    \subsubsection{证明$H(X|Y)=H(X,Y)-H(Y)$}
    \begin{align}
        H(X|Y)&=\sum p(x,y)\log_2\frac{1}{p(x|y)}\\
        H(X,Y)-H(Y)&=\sum p(x,y)\log_2\frac{1}{p(x,y)}-\sum p(y)\log_2\frac{1}{p(y)}\\
        &=\sum p(y)\sum p(x|y)\log_2\frac{1}{p(x,y)}-\sum p(y)\log_2\frac{1}{p(y)}\\
        &=\sum p(y)(\sum p(x|y)\log_2\frac{1}{p(x,y)}-\log_2\frac{1}{p(y)})\\
        &=\sum p(y)(\sum p(x|y)\log_2\frac{1}{p(x,y)}-\sum p(x|y)\log_2\frac{1}{p(y)})\\
        &=\sum p(y)\sum p(x|y)\log_2\frac{p(y)}{p(x,y)}\\
        &=\sum p(y)\sum p(x|y)\log_2\frac{1}{p(x|y)}\\
        &=\sum p(x,y)\log_2\frac{1}{p(x|y)}\\
        &=H(X|Y)
    \end{align}
    \subsection{互信息}
    \begin{align}
        I(X;Y)&=\sum_{i,j}p(x_i,y_j)\log_2\frac{p(x_i|y_j)}{p(x_i)}\\
        &=\sum_{i,j}p(x_i,y_j)\log_2\frac{p(x_i,y_j)}{p(x_i)p(y_j)}\\
        &=\sum_{i,j}p(x_i,y_j)\log_2\frac{p(y_j|x_i)}{p(y_j)}\\
        &=I(Y;X)
    \end{align}
    \paragraph{互信息的一些性质:}
    \begin{enumerate}
        \item X || Y 时,$I(X;Y)=0$
        \item X = Y 时, $I(X;Y)=H(X)=H(Y)$
        \item X = f(Y) 时, $I(X;Y)=H(X)$
    \end{enumerate}
    \paragraph{互信息有如下表达形式:}
    \begin{align}
        I(X;Y)&=H(X)+H(Y)-H(X,Y)\\
        &=H(X)-H(X|Y)\\
        &=H(Y)-H(Y|X)
    \end{align}
    \paragraph{证明$I(X;Y)=H(X)+H(Y)-H(X,Y)$\\}
    有如下公式成立:
    \[p(x_i)=\sum_{y\in Y}p(x,y)\]
    \begin{align}
        I(X;Y)&=H(X)+H(Y)-H(X,Y)\\
        &=\sum p(x)\log_2\frac{1}{p(x)}+\sum p(y)\log_2\frac{1}{p(y)}-\sum p(x,y)\log_2\frac{1}{p(x,y)}\\
        &=\sum p(x,y)\log_2\frac{1}{p(x)}+\sum p(x,y)\log_2\frac{1}{p(y)}-\sum p(x,y)\log_2\frac{1}{p(x,y)}\\
        &=\sum p(x,y)\log_2\frac{p(x,y)}{p(x)p(y)}
    \end{align}
    \paragraph{证明$I(X;Y)=H(X)-H(X|Y)$\\}
    \begin{align}
        I(X;Y)&=H(X)-H(X|Y)\\
        &=\sum p(x)\log_2\frac{1}{p(x)}-\sum p(x,y)\log_2\frac{1}{p(x|y)}\\
        &=\sum p(x,y)\log_2\frac{p(x|y)}{p(x)}
    \end{align}
    \subsection{Chain Rule}
    二元的情况下有$H(X,Y)=H(X|Y)+H(Y)$\\

    对于三元的情况$H(X_1,X_2,X_3)$,设$(X_2,X_3)=Z$,则有
    \begin{align}
        H(X_1,X_2,X_3)&=H(X_1,Z)\\
        &=H(X_1|Z)+H(Z)\\
        &=H(X_1|X_2,X_3)+H(X_2|H_3)+H(X_3)
    \end{align}

    推广到n元的情况下,得到$H(X_1,...,X_n)=\sum_{i=1}^nH(X_i|X_{i+1},...,X_n)$
    \subsection{Conditional Matual Information}
\end{document}